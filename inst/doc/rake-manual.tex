\documentclass[12pt]{TDH-article}
%\VignetteIndexEntry{Rake Manual}
\usepackage{Sweave}
\begin{document}

\title{Rake Manual}
\author{Toby Dylan Hocking}
\maketitle

\newcommand{\rake}{Rake}

\section{Introduction}
\rake\ is an R package that allows you to easily perform Rake analysis \cite{Lohr:rake}.

This manual teaches how to use the \rake\ package by walking you through a tutorial rake analysis on the data that is provided with the package. For a more thorough overview of the \rake\ functions and their arguments, please refer to the package documentation pages. For a more detailed discussion of the theory, refer to \cite{Lohr:rake}.

\section{Tutorial}
For the tutorial analysis, we use the ASA certification data included
with the \rake\ package, which is the same data as described in
\cite{Lohr:certify}.

A standard \rake\ analysis consists of 3 steps:
\be
\item Creating the {\tt "rake"} class object using the {\tt rake} function.
\item Performing the rake weight adjustment using the {\tt rakeadj} function.
\item Obtaining new sample weights and reweighted data  using the {\tt predict.rake} function.
\ee

These 3 steps have been wrapped into a function called {\tt simpleRake}.
This effectively turns a 3-step analysis into a 1-step analysis, but should
only be used if the {\tt "rake"} class object's breakdown of sample weights by
class is of no interest.

\subsection{{\tt rake}}
The {\tt "rake"} class object contains a summary of sample weights, broken down by classes for 2 categorical variables of interest.

\begin{Schunk}
\begin{Sinput}
> library(rake)
> data(certify)
> certify$COLLEGE[certify$COLLEGE != "P"] <- "N"
> certify$WORKENV[!certify$WORKENV %in% c("I", "A")] <- "O"
> r <- rake(certify, "COLLEGE", "WORKENV")
> print(r)
\end{Sinput}
\begin{Soutput}
  data: certify
rowvar: WORKENV
colvar: COLLEGE
     N    P     
A  434 1787 2221
I 1011  798 1809
O  520  451  971
  1965 3036 5001
\end{Soutput}
\end{Schunk}

The idea behind raking is to re-weight the sample so that the marginal total sample weights, as summarized in the {\tt "rake"} class object, are equivalent to the known marginal total weights of the population.

\subsection{{\tt rakeadj}}
This re-weighting is accomplished with the {\tt rakeadj} function:

\begin{Schunk}
\begin{Sinput}
> r <- rakeadj(r, statpoptotal, TRUE)
\end{Sinput}
\begin{Soutput}
The rake adjustment converged in 20 steps.
\end{Soutput}
\begin{Sinput}
> print(r)
\end{Sinput}
\begin{Soutput}
  data: certify
rowvar: WORKENV
colvar: COLLEGE
          N         P         
A  361.8412 1338.4826 1700.324
I  848.6274  601.7674 1450.395
O 1039.9679  810.3134 1850.281
  2250.4366 2750.5634 5001.000
\end{Soutput}
\begin{Sinput}
> print(r/sum(r) * sum(statpoptotal$weight[statpoptotal$name == 
+     rownames(r)]))
\end{Sinput}
\begin{Soutput}
  data: certify
rowvar: WORKENV
colvar: COLLEGE
         N         P      
A 1346.431  4980.569  6327
I 3157.790  2239.210  5397
O 3869.779  3015.221  6885
  8374.000 10235.000 18609
\end{Soutput}
\end{Schunk}

Note that in this dataset, we rake on the 2 categorical variables {\tt "WORKENV"} and {\tt "COLLEGE"} because we know that the population values for these variables:

\begin{Schunk}
\begin{Sinput}
> print(statpoptotal)
\end{Sinput}
\begin{Soutput}
  name weight
1    A   6327
2    I   5397
3    O   6885
4    P  10235
5    N   8374
\end{Soutput}
\end{Schunk}

The {\tt "rake"} class object maintains the weight sum of the sample, but its components are adjusted such that the marginal total sample weights now have the same ratio as those in the population.

\subsection{{\tt predict.rake}}
To obtain the corresponding re-weighted unit weights for the original sample data, use the {\tt predict.rake} function:

\begin{Schunk}
\begin{Sinput}
> for (i in 1:5) {
+     cat("Variable:", names(certify)[i], "\n")
+     data <- summary(as.factor(certify[, i]))
+     pred <- predict(r, i, forcefactor = TRUE)$data.est
+     print(rbind(data, pred))
+     cat("\n")
+ }
\end{Sinput}
\begin{Soutput}
Variable: CERTIFY 
            0        1        2        3        4        5
data 12.00000 1321.000 1114.000 269.0000 337.0000 1948.000
pred 14.64323 1288.808 1114.226 272.1095 345.7145 1965.500

Variable: APPROVE 
            0        1        2        3        4        5
data 102.0000 766.0000 1025.000 820.0000 470.0000 1818.000
pred 108.3808 727.1895 1023.779 822.4286 468.2622 1850.960

Variable: SPECCERT 
            0       1        2        3        4        5
data 63.00000 552.000 901.0000 451.0000 488.0000 2546.000
pred 67.03614 540.527 913.3438 464.5248 491.8947 2523.673

Variable: WOULDYOU 
            0        1        2        3        4        5
data 42.00000 1566.000 1119.000 212.0000 573.0000 1489.000
pred 45.91886 1511.456 1113.057 203.1978 600.5737 1526.797

Variable: RECERT 
            0        1        2        3        4        5
data 92.00000 951.0000 936.0000 596.0000 404.0000 2022.000
pred 94.25057 921.3488 924.8003 598.8645 410.3751 2051.361
\end{Soutput}
\end{Schunk}

This function returns a list of three elements:
\bd
\ti[weight] Vector of re-weighted sample weights, in the same order as the sample data's rows.
\ti[data] Vector of sample data, in the same order as the sample data's rows.
\ti[data.est] Vector of sample data with the new values after re-weighting. For categorical data, this is a factor summary. For numeric data, this is a vector of re-weighed data, in the same order as the sample data's rows.
\ed

\subsection{{\tt simpleRake}}
This wraps the raking process inside a single function:

\begin{Schunk}
\begin{Sinput}
> example(simpleRake)
\end{Sinput}
\begin{Soutput}
smplRk> data(certify)

smplRk> # Merge some classes since we have limited population data
smplRk> # We define classes of COLLEGE as P or not P = N
smplRk> certify$COLLEGE[   certify$COLLEGE  !=    "P"      ] <- "N"

smplRk> # We define classes of WORKENV to be I, A, or else = O
smplRk> certify$WORKENV[ ! certify$WORKENV %in% c("I","A") ] <- "O"

smplRk> rakeresult <- simpleRake(certify, statpoptotal, "WORKENV", "COLLEGE",
smplRk+ "WOULDYOU", TRUE)

smplRk> data <- rakeresult$data

smplRk> pred <- rakeresult$data.est

smplRk> print(rbind(data,pred))
            0        1        2        3        4        5
data 42.00000 1566.000 1119.000 212.0000 573.0000 1489.000
pred 45.91886 1511.456 1113.057 203.1978 600.5737 1526.797
\end{Soutput}
\end{Schunk}

Note that you don't get access to the {\tt "rake"} class object. This is usually used if you are only interested in the sample weights after re-weighting.

\section{Conclusion}
\rake\ is a package that implements the robust technique of sample re-weighting based on the raking technique.

The 3-step raking should be used for most applications to get a better feel for the sample data class weights, but the 1-step {\tt simpleRake} can be performed if only the unit sample weights are needed.

\appendix
\bibliographystyle{plain}
\bibliography{TDH-MCB}

\end{document}
